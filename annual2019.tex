\position{
%%%%%%%%%%%%%%%%%%%%%%%%%%%%%%%%%%%%%%%%%%%%%%%%%%%%%%%%%%%%%%%%%%%%%%%%%%%
% 2019年度 東大数理における該当する身分を選択し、それ以外の項目は削除
% して下さい。
%%%%%%%%%%%%%%%%%%%%%%%%%%%%%%%%%%%%%%%%%%%%%%%%%%%%%%%%%%%%%%%%%%%%%%%%%%%
特任研究員 (Project Researcher)
%%%%%%%%%%%%%%%%%%%%%%%%%%%%%%%%%%%%%%%%%%%%%%%%%%%%%%%%%%%%%%%%%%%%%%%%%%%
}


%%%%%%%%%%%%%%%%%%%%%%%%%%%%%%%%%%%%%%%%%%%%%%%%%%%%%%%%%%%%%%%%%%%%%%%%%%%%
% 氏名(ローマ字綴りは名字は全て大文字,名前は最初の字だけ大文字)
%%%%%%%%%%%%%%%%%%%%%%%%%%%%%%%%%%%%%%%%%%%%%%%%%%%%%%%%%%%%%%%%%%%%%%%%%%%%
% 日本人
% \msname{TSUBOI Takashi}{坪井 俊}{つぼいたかし}
% 日本人以外(漢字表記なし)
% \msname*{SUTHICHITRANONT Noppakhun}{スッティチトラノン ノッパクン}{すってぃちとらのん}
% 日本人以外(漢字表記あり
% \msname*{LI Xiaolong}{リ シャオロン}[李 曉龍]{りしゃおろん}
%%%%%%%%%%%%%%%%%%%%%%%%%%%%%%%%%%%%%%%%%%%%%%%%%%%%%%%%%%%%%%%%%%%%%%%%%%%
\msname{TAKAHASHI Kazune}{高橋 和音}{たかはし かずね}
%学生で学振DC1または学振DC2に該当する者はコメントを外してください。
%\mscourse{(学振DC1)}
%\mscourse{(学振DC2)}
%学生でFMSPコース生に該当する者はコメントを外してください。
%\mscourse{(FMSPコース生)}

\section{A. 研究概要}

\begin{enumerate}
  \item {\bf Homogeneous H\'enon type elliptic equations with critical Sobolev growth~[2]~[3]} \\
  I worked on the following
  homogeneous H\'enon type elliptic equations with critical Sobolev growth:
  $- \Delta u = \lambda \Psi u + \lvert x \rvert^\alpha u^{2^*-1}$,
  where
  $\Psi$ is a given non-trivial bounded function which may vanish on the bounded boundary. I proved that there exists a positive solution if $\alpha > 0$ is sufficiently small for the dimension $N \geq 4$ by developing new families of test Talenti functions properly. It is known that there is another result for $N = 3$. Thus this seems to be the best result for this type of estimates.
  \item {\bf Kirchhoff-H\'enon type equations with critical Sobolev growth~[1]~[2]} \\
  I applied these new families of functions above for the following Kirchhoff type equations with critical Sobolev growth:
  $- \left( a + b \left\| Du \right\|^{p-2}_{L^2(\Omega)} \right) \Delta u = \Psi u^{q-1} + \lvert x \rvert^\alpha u^{2^* - 1}$.
  Kirchhoff-H\'enon type equations had not been ever studied before my work. In addition, though the standard Kirchhoff type equations are for the case $\alpha = 0$ and $p = 4$, my study covers the case $\alpha > 0$ and $2 < p < q < 2^*$ for the dimension $N \geq 3$. I introduced a new method to obtain estimates of the minimax energy where the fibering map cannot be explicitly represented, and proved the existence of a positive solution for small $\alpha \geq 0$.
  \item {\bf Stand wave solutions of
  nonlinear Schr\"{o}dinger-Poisson systems~[9]} \\
  This is a joint work with Hiroyuki Miyahara (UTokyo).
  We worked on stand wave solutions
  of the following nonhomogeneous
  nonlinear Schr\"{o}dinger-Poisson systems:
  $-\Delta u -a \phi \left\lvert u \right\rvert^{q-1} u = \lambda u
  + b \left\lvert u \right\rvert^{p-1} u, -\Delta \phi =
  \left\lvert u \right\rvert^{q+1}$.
  There were many previous studies for the case the dimension $N = 3$,
  but our study covered the case $N \geq 3$.
  We proved existence and nonexistence theorem
  where each $p, q+1$ is critical or subcritical.
  Especially, for some specific case,
  we determined the range of $\lambda$ where a non-trivial solution
  $(u, \phi)$ does exist or does not exist.
  \item {\bf Generalized Joseph-Lundgren exponent~[5]} \\
  This is a joint work with Prof.~Yasuhito Miyamoto (UTokyo).
  We worked on the following ordinary differential equation for $r
  \in (0, \infty)$:
  $r^{-(\gamma-1)} (r^\alpha \lvert u' \rvert^{\beta -1 } u')'
  + \lvert u \rvert^{p-1} u = 0$.
  Here the left term represents
  a generalized radial differential operator
  that covers, for example, the $N$-dimensional
  usual Laplacian, $m$-Laplacian or $k$-Hessian.
  In previous research
  the generalized Joseph-Lundgren exponent for this operator
  was calculated, but there was a technical lowerbound for
  the exponent $p$.
  We removed that bound by transforming the equation
  and determined intersection numbers
  which role differently on $p$.
  \item {\bf Nonhomogeneous semilinear elliptic equations involving
  critical Sobolev exponent~[6]~[10]} \\
  I worked on the following
  nonhomogeneous semilinear elliptic equation
  involving the critical Sobolev exponent:
  $-\Delta u + a u = b u^{2^* - 1} + \lambda f$.
  I proved that provided
  $b$ achieves its maximum at an inner point of the
  domain and $a$ has a growth of the exponent $q$
  in some neighborhood of that point, then
  if the dimension of the domain is less than $6 + 2q$,
  there exist at least two positive solutions.
  It seems to be new that the coefficient of a linear term affects
  the dimension of the domain on which solutions exist.
  \item {\bf Zero-dimensional fold and cut~[4]~[7]} \\
  This is a joint work with
  Yasuhiko Asao (UTokyo), Prof.~Erik D.~Demaine (MIT),
  Prof.~Martin L.~Demaine (MIT), Hideaki Hosaka (Azabu high school),
  Prof.~Akitoshi Kawamura (UTokyo)
  and Prof.~Tomohiro Tachi (UTokyo).
  We showed how to fold a piece of paper and punch one hole
  on given $n$ points
  so as to produce any desired patterns of holds.
  There is $4$ variants of problems;
  the paper is finite or infinite
  and we allow or forbid the crease on the points.
  In~[7], we gave solutions for each case and the order of crease
  are bounded on the polynomial order of $n$ and the paper ratio
  $r$.
  In the sequel paper~[4], we also gave a definition of
  the complexity of folds, which
  will be useful for further studies that determine
  NP-hardness of complex folding problems.
  \item {\bf Control model for traffic lights~[8]} \\
  This is a joint work with Xinchi Huang (UTokyo).
  We worked on discrete model of traffic lights which would not
  cause traffic jams.
  An observation data showed each number of cars for the pair of
  inlet and outlet of roads but there was ambiguity of
  the route of each cars.
  We let the problem come to
  $n$-varieties transportation problem but it is known as
  NP-complete. Therefore we also suggested an algorithm
  that superimpose usual max flow problems.
  \item {\bf Application of SAT-solver for AI} \\
\end{enumerate}

\section{B. 発表論文}
%%%%%%%%%%%%%%%%%%%%%%%%%%%%%%%%%%%%%%%%%%%%%%%%%%%%%%%%%%%%%%%%%%%%%%%%%%%%%%
% 5年以内(2015〜2019年度)10篇まで書いて下さい。但し、2019年1月1日〜
% 2019年12月31日に出版されたものは、10篇を超えてもすべて含めて下さい。
% 様式は以下の例のように
% 著者・共著者名・題名・ジャーナル名・巻・年・ページの順に書いて下さい.
% タイトルの前に著者・共著者名を入れる形です。
% 共著の場合 T. Katsura and #.####などと書きwith 共著者名とはしない様に
% お願い致します。
%%%%%%%%%%%%%%%%%%%%%%%%%%%%%%%%%%%%%%%%%%%%%%%%%%%%%%%%%%%%%%%%%%%%%%%%%%%%%
%\begin{enumerate}
%\sfcode`\.1000 % この行は消さないで
%\item G. van der Geer and T. Katsura:``On a stratification of
%the moduli of K3 surfaces'',
%J. Eur. Math. Soc. {\bfseries 2} (2000) 259--290.
%\end{enumerate}

\begin{enumerate}
  \item[] {\bf Refereed Papers}
  \sfcode`\.1000 % この行は消さないで
  \item Kazune Takahashi: ``Positive solutions of Kirchhoff-Hénon type elliptic equations with critical Sobolev growth'', to appear in Topological Methods in Nonlinear Analysis.
  \item Kazune Takahashi: ``Hénon type elliptic equations with critical Sobolev growth'', Doctor Dissertation, The University of Tokyo (2019). {\it Based on [1] and [3].}
  \item Kazune Takahashi: ``Positive solution for an Hénon type equation with critical Sobolev growth'', Electronic Journal of Differential Equations {\bf 2018(194)} (2018) 1--17.
  \item Yasuhiko Asao, Erik Demaine, Martin Demaine, Hideaki Hosaka, Akitoshi Kawamura, Tomohiro Tachi and Kazune Takahashi: ``Folding and Punching Paper'', Journal of Information Processing {\bf 25} (2017) 590--600.
  \item Yasuhito Miyamoto and Kazune Takahashi: ``Generalized Joseph-Lundgren exponent and intersection properties for supercritical quasilinear elliptic equations'', Archiv der Mathematik {\bf 108(1)} (2017) 71--83.
  \item Kazune Takahashi: ``Semilinear elliptic equations with critical Sobolev exponent and non-homogeneous term'', Master Thesis, The University of Tokyo (2015).
  \item[] {\bf Refereed Conference Abstracts}
  \item Yasuhiko Asao, Erik Demaine, Martin Demaine, Hideaki Hosaka, Akitoshi Kawamura, Tomohiro Tachi and Kazune Takahashi: ``Folding and Punching Paper'', Abstracts from the 19th Japan Conference on Discrete and Computational Geometry, Graphs and Games (2016) 40--41.
  \item[] {\bf Non-Refereed Papers}
  \item Xinchi Huang, Kazune Takahashi, Yasuhisa Kishi, Masahiko Kanai, and Takafumi Mase: ``A modified model on a traffic network and signal optimization'', Suurikagaku Jissenkenkyu Letter, LMSR {\bf 2018(4)} (2018) 1--5.
  \item[] {\bf Preprints}
  \item Hiroyuki Miyahara and Kazune Takahashi: ``Existence and Nonexistence of Standing Wave Solutions of Nonlinear Schr\"{o}dinger-Poisson System'', preprint.
  \item[] {\bf Misc}
  \item Kazune Takahashi: ``Semilinear elliptic equations with critical Sobolev exponent and non-homogeneous term'',RIMS K\^{o}ky\^{u}roku {\bf 2006} (2016) 1--11.
\end{enumerate}


\section{C. 口頭発表}
%%%%%%%%%%%%%%%%%%%%%%%%%%%%%%%%%%%%%%%%%%%%%%%%%%%%%%%%%%%%%%%%%%%%%%%%%%%%%%%
% シンポジュームや学外セミナーでの発表で5年以内(2015〜2019年度)10項目まで。
% タイトル・シンポジューム(またはセミナー等)名・場所・月・年を
% 書いて下さい.国際会議の場合は国名をお願いします.タイトルは原題で。
%%%%%%%%%%%%%%%%%%%%%%%%%%%%%%%%%%%%%%%%%%%%%%%%%%%%%%%%%%%%%%%%%%%%%%%%%%%%%%%
%\begin{enumerate}
%\item (1) 曲面の写像類群とは, (2) 写像類群をめぐるこれまでの結果と夢,
%Encounter with Mathematics 第11回, 中央大学理工学部,
%1999年4,5月.
%\end{enumerate}

\begin{enumerate}
  \item[] {\bf International Conferences}
  \item (With Yasuhiko Asao, Erik Demaine, Martin Demaine, Hideaki Hosaka, Akitoshi Kawamura and Tomohiro Tachi) Folding and Punching Paper, The 19th Japan Conference on Discrete and Computational Geometry, Graphs, and Games, Tokyo University of Science, Japan, Sep 2016.
  \item Semilinear elliptic equations with critical Sobolev exponent and non-homogeneous term, RIMS Workshop: Shapes and other properties of solutions of PDEs, RIMS, Kyoto University, Japan, Nov 2015. [Invited]
  \item[] {\bf Domestic Conferences}
  \item Existence and Nonexistence of Standing Wave Solutions of Nonlinear Schr\"{o}dinger-Poisson System, The 39th Differential Equation Seminar at Yokohama National University, Yokohama National University, Japan, Aug 2016. [Invited]
\end{enumerate}

\section{G. 受賞}
%%%%%%%%%%%%%%%%%%%%%%%%%%%%%%%%%%%%%%%%%%%%%%%%%%%%%%%%%%%%%%%%%%%%%%%%%%
% 過去5年の間にありましたら書いて下さい。
%%%%%%%%%%%%%%%%%%%%%%%%%%%%%%%%%%%%%%%%%%%%%%%%%%%%%%%%%%%%%%%%%%%%%%%%

\begin{enumerate}
  \item[] {\bf Awards}
  \item (With Yasuhiko Asao, Erik Demaine, Martin Demaine, Hideaki Hosaka, Akitoshi Kawamura and Tomohiro Tachi) Folding and Punching Paper, Specially Selected Paper, Information Processing Society of Japan, Aug 2017.
  \item[] {\bf International Programming Contests}
  \item SamurAI Coding 2014--15, World Final: 6th place, 77th Information Processing Society of Japan National Convention, Kyoto University, Japan, Mar 2015.
  \item[] {\bf Domestic Programming Contests}
  \item Code Runner 2015, Final Round: 1st place, Recruit Career, Tokyo, Dec 2015.
  \item Code Runner 2014, Final Round: 7th place, Recruit Career, Tokyo, Nov 2014.
  \item Code Festival 2014 AI Challenge, Final Round: 3rd place, Recruit Holdings, Tokyo, Nov 2014.
\end{enumerate}
